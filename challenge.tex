\learningobjective{At the end of this challenge, the scholar will be able to explain the concept of 	exttt{std::queue} as a FIFO data structure in C++, including its advantages over manually implemented queues in C and its use cases for sequential data processing.}
\begin{challenge}
    \chatitle{FIFO Data Structures with std::queue in C++}
    \begin{chadescription}
    In C++, the \texttt{std::queue} class implements a First-In-First-Out (FIFO) data structure, where elements are added to the back and removed from the front. 
    This structure is ideal for scenarios where the order of operations or data processing must be preserved, such as task scheduling or buffering.

    Unlike arrays or other containers, \texttt{std::queue} provides restricted access to its elements, allowing interaction only with the front and back. 
    This restriction enforces the FIFO principle and simplifies usage by focusing on the operations of insertion (\texttt{push()}) and removal (\texttt{pop()}). 
    Additionally, \texttt{std::queue} is a container adapter, meaning it uses other containers like \texttt{std::deque} or \texttt{std::list} as its underlying storage.

    In this challenge, you will explore how to use \texttt{std::queue} to model real-world FIFO behavior and understand its advantages over manually implementing queues in C. 
    By the end, you will have a clear understanding of when and why to use \texttt{std::queue} in C++ programs.
    \end{chadescription}

    \begin{task}
        Write a simple program using \texttt{std::queue} that performs the following steps:
        \begin{enumerate}
            \item Declare an \texttt{std::queue} to store integers.
            \item Add the numbers \texttt{10}, \texttt{20}, and \texttt{30} to the queue using \texttt{push()}.
            \item Print the front element of the queue using \texttt{front()}.
            \item Remove the front element using \texttt{pop()}.
            \item Print the new front element and check if the queue is empty using \texttt{empty()}.
        \end{enumerate}

        Save your program and run it to ensure the output matches your expectations.

        \begin{questions}
            \item How does the restricted access provided by \texttt{std::queue} simplify its usage compared to manually implementing a queue in C?
            \item What happens if you call \texttt{pop()} on an empty queue? How can you prevent such errors?
            \item Why is it important for the \texttt{empty()} method to check the queue’s state before performing operations?
        \end{questions}
    \end{task}

    \begin{task}
        Write a program using \texttt{std::queue} that simulates a customer service system:
        \begin{enumerate}
            \item Declare an \texttt{std::queue} to store customer names as strings.
            \item Add five customer names to the queue using \texttt{push()}.
            \item Implement a loop that serves each customer by:
                \begin{itemize}
                    \item Printing the name of the customer being served using \texttt{front()}.
                    \item Removing the customer from the queue using \texttt{pop()}.
                \end{itemize}
            \item Print a message when all customers have been served, using \texttt{empty()} to check the queue.
        \end{enumerate}

        Save your program, compile it, and ensure it runs correctly.

        \begin{questions}
            \item How does the loop in this task demonstrate the FIFO behavior of \texttt{std::queue}?
            \item Why is the queue’s restricted access (only to the front and back) advantageous in this scenario?
            \item What are some real-world situations where a FIFO structure like \texttt{std::queue} would be useful?
        \end{questions}
    \end{task}

    \begin{task}
        Write a program using \texttt{std::queue} that processes a sequence of numbers as follows:
        \begin{enumerate}
            \item Declare an \texttt{std::queue} to store integers.
            \item Read a sequence of 10 integers from the user and add them to the queue using \texttt{push()}.
            \item Implement a function named \texttt{processQueue} that:
                \begin{itemize}
                    \item Doubles each number in the queue.
                    \item Removes the processed number from the queue using \texttt{pop()}.
                \end{itemize}
            \item Print the doubled numbers as they are processed.
        \end{enumerate}

        Save your program, compile it, and ensure it runs correctly. This task requires you to think carefully about how to interact with the queue dynamically.

        \begin{questions}
            \item How does the \texttt{processQueue} function demonstrate dynamic interaction with an \texttt{std::queue}?
            \item Why is \texttt{std::queue} a good choice for processing sequential input data?
            \item How could you extend this program to process more complex operations on the queue’s elements?
        \end{questions}
    \end{task}

    \begin{advice}
        Working with \texttt{std::queue} helps you understand the simplicity and power of FIFO data structures in C++. 
        Unlike manually implemented queues in C, \texttt{std::queue} provides built-in methods that make managing sequential data easy and efficient. 
        Through this challenge, you have explored how to add, remove, and process elements while preserving their order. 
        Always remember to choose \texttt{std::queue} when you need a straightforward way to manage sequential operations where the order of processing matters. 
        With practice, you'll find \texttt{std::queue} to be an excellent tool for solving many common programming problems.
    \end{advice}
\end{challenge}
